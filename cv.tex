\documentclass[margin, 10pt]{res} % Use the res.cls style, the font size can be changed to 11pt or 12pt here

% \usepackage{helvet} % Default font is the helvetica postscript font
%\usepackage{newcent} % To change the default font to the new century schoolbook postscript font uncomment this line and comment the one above
\renewcommand{\familydefault}{\sfdefault}
\usepackage{enumitem}
\setlist[itemize]{leftmargin=*}
\setlength{\textwidth}{5.1in} % Text width of the document

\begin{document}

%----------------------------------------------------------------------------------------
%	NAME AND ADDRESS SECTION
%----------------------------------------------------------------------------------------

\moveleft.5\hoffset\centerline{\large\bf Gregory Boyd} % Your name at the top
 
\moveleft\hoffset\vbox{\hrule width\resumewidth height 1pt}\smallskip % Horizontal line after name; adjust line thickness by changing the '1pt'
 
\moveleft.5\hoffset\centerline{gregory.boyd@merton.ox.ac.uk} % Your address
\moveleft.5\hoffset\centerline{(+44) 07531542501}

%----------------------------------------------------------------------------------------

\begin{resume}



\section{Education}

\begin{itemize}[label={\hspace{0.21cm}}]
    \item \textbf{University of Oxford, Merton College} \\
    \textbf{DPhil, Third Year, Anticipated 2025} \\
    Quantum Algorithms in the Near and Future Term \\
    Supervisors: Simon Benjamin and Bálint Koczor
    \item \textbf{University of Cambridge, Emmanuel College} \\
    \textbf{Master in Science, Awarded 2021} \\
    First Class Honors in all years and awarded a Davies
    Scholarship and College Prize for performance in examinations.
    Relevant courses include: Quantum Information, Theory of
    Quantum Matter, Quantum Field Theory, Information Theory
\end{itemize}

\section{Research \\Experience}

\begin{itemize}
    \item Parallelization of key quantum subroutines, producing a net
    savings in computational space-time volume \\
    \textbf{Low-Overhead Parallelisation of LCU via Commuting
    Operators} \\ (arXiv 2023)
    \item Research into optimization techniques for variational quantum
    algorithms, leading to a quantum-aware re-expression of the
    problem with improved convergence. \\
    \textbf{Training Variational Quantum Circuits with CoVaR:
    Covariance Root Finding with Classical Shadows}
    (Gregory Boyd and Bálint Koczor, PRX 2022)
    \item Researched Quantum Topological Data Analysis during an
    internship as a research scientist at \textbf{Quantinuum}, working in a
    small team starting up a workstream of which I became a vital
    member, providing significant circuit optimizations, and resulting in a patent.
    \item Freelance consultant for \textbf{Quantum Motion}, collaborating in the work \\
    \textbf{Low Depth Phase Oracle Using a Parallel Piecewise Circuit} (arXiv 2024)
    \item Investigation and regularisation of unprecedentedly large subspace expansions. \\
    \textbf{High-Dimensional Subspace Expansion Using Classical Shadows} (arXiv 2024)

    \item Resource estimation of subroutines for a practical
    implementation of early fault-tolerant quantum simulation,
    allowing for useful hardware requirements for the earliest
    quantum algorithms.
    \item Modelling superconducting qubits at the hardware level, involving
    modifying and testing a state-of-the-art GPU-solver on
    the Cambridge Computing Cluster, developing an in-depth
    knowledge of modern quantum computing architectures and
    computational techniques.
\end{itemize}

\section{Programming \\and IT Skills}

\begin{itemize}
    \item Developed simulations for complex physical systems using
    Python and C++, including a simulator for optimal control of
    superconducting qubits from the ground up in C++ and various
    many body simulations, and analyzing the results to provide
    meaningful comparisons with theory.
    \item Used Mathematica to produce simulations of a complex quantum variational optimization scheme. 
    \item Used HPC resources to do large scale simulations of early
    quantum computers.
    \item Familiar with a variety of quantum simulators/software packages,
    including Cirq, Qiskit, QuEST, Q\# and OpenFermion.
    \item Git proficiency, used in professional and personal projects
    \item Developed an Android e-reader/audiobook player app.
    \item Developed a WebAssembly plugin providing a full LaTeX compiler with extra utilities for the note-taking app Obsidian.
\end{itemize}


\section{Problem Solving Skills}

\begin{itemize}
    \item Worked as part of small teams, producing and presenting
    solutions to problems and co-developing solutions with others.
    This has taught me to provide and utilize support, incorporating a
    variety of ideas to solve a problem.
    \item Analyzed datasets using statistical methods to illustrate key
    properties and trends, allowing for identification of problem
    features and routes to solutions.
    \item Studied a diverse range of subjects including chemistry and
    materials science. This broad grounding has given me
    experience in adapting methods for use in varied contexts.
\end{itemize}


\section{Communication Skills}

\begin{itemize}
    \item Experience in public speaking through presenting research at the
    Quantum Computing Hub Project Forum, and Quantum
    Algorithms for Chemistry conference at CECAM
    \item Presented a highlighted poster at QIP 2023, and received an honourable mention for the SEEQA 2024 poster prize.
    \item Oxford Tutor, providing small group teaching to Undergraduates
    at a world-class university.
    \item Tutored pupils at A level with differing degrees of aptitude, clearly
    communicating knowledge at the correct level to provide marked
    improvements in results.
    \item Outreach to schools on behalf of University Access.
\end{itemize}


\end{resume}

\vspace{3cm}
Reference: Simon Benjamin, simon.benjamin@materials.ox.ac.uk

\end{document}